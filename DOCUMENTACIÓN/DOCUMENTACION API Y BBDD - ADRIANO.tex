\documentclass[12pt]{article}
\usepackage[spanish]{babel}
\usepackage[letterpaper,top=2cm,bottom=2cm,left=3cm,right=3cm,marginparwidth=1.75cm]{geometry}
\usepackage{tabularx}
\usepackage{fancyvrb}
\usepackage{graphicx}
\usepackage{setspace}
\usepackage{ragged2e}
\usepackage[T1]{fontenc}
\usepackage{librefranklin}

\renewcommand*\familydefault{\sfdefault}
\DefineVerbatimEnvironment{Json}{Verbatim}{fontsize=\small, frame=single, framerule=0.5pt, framesep=5pt}
\setlength{\parindent}{0pt}
\onehalfspacing
\setcounter{tocdepth}{2}


\begin{document}

\begin{titlepage}

    \centering
    \vspace*{1 cm}
    \Huge
    \textbf{Dockerización de API y BBDD}

    \vspace{0.5 cm}
    \Large
    Programación de Servicios y Procesos

    \vspace{5.5 cm}
    \textbf{Adrián Condines Celada}

    \vspace{0.8 cm}    
    \normalsize
    Aula Estudio\\

    \vspace{0.8 cm}
    2º Ciclo Superior - Desarrollo de Aplicaciones Multiplataforma\\

    \vspace{0.8 cm}
    Curso 2025 - 2026

\end{titlepage}

\tableofcontents
\newpage

\section{INTRODUCCIÓN Y OBJETIVOS}

\section{API}

\justify
Está sección detalla los endpoints disponibles en la API para gestionar los usuarios y los grupos.

\subsection{Listar todos los usuarios}

\subsubsection{Descripción}

\justify
Obtiene una lista de todos los registros almacenados en la tabla \texttt{users}.

\begin{itemize}

    \item \textbf{Método:} GET
    \item \textbf{URL:} \texttt{/api/usuarios}
    
\end{itemize}

\subsubsection{Ejemplo de Input}

\justify
No requiere ningún input (ni parámetros en la URL ni body).

\subsubsection{Ejemplo de Output (Éxito)}

\justify
Responde con un array de objetos JSON, cada uno representando un rapero.

\begin{Json}

[
    {
        "id": 1,
        "name": "Bruno"
    },
    {
        "id": 2,
        "name": "Adriano"
    }
]

\end{Json}

\subsubsection{Ejemplo de Output (Error)}

\justify
Responde con un JSON incluyendo un mensaje de error.

\begin{Json}

{
  "error": "Error al devolver los datos de los usuarios: (error)"
}

\end{Json}

\subsection{Listar un usuario en concreto}

\subsubsection{Descripción}

\justify
Obtiene un solo registro en la tabla \texttt{users}.

\begin{itemize}

    \item \textbf{Método:} GET
    \item \textbf{URL:} \texttt{/api/usuarios/:id}
    
\end{itemize}

\subsubsection{Ejemplo de Input}

\justify
Parámetro \texttt{id} en la URL.

\subsubsection{Ejemplo de Output (Éxito)}

\justify
Responde con un JSON en el que se incluye el usuario a buscar.

\begin{Json}

[
    {
        "id": 1,
        "name": "Bruno"
    }
]

\end{Json}

\subsubsection{Ejemplo de Output (Error)}

\begin{Json}

\justify
Responde con un JSON incluyendo un mensaje de error.

{
  "error": "Error al devolver los datos del usuario: (error)"
}

\end{Json}

\subsection{Añadir un usuario}

\subsubsection{Descripción}

\justify
Añade un registro a la tabla \texttt{users}.

\begin{itemize}

    \item \textbf{Método:} POST
    \item \textbf{URL:} \texttt{/api/usuarios}
    
\end{itemize}

\subsubsection{Ejemplo de Input}

\justify
Un objeto JSON en el body de la petición con el nombre del nuevo usuario.

\begin{Json}

{
    "name": "Adriano"
}

\end{Json}

\subsubsection{Ejemplo de Output (Éxito)}

\justify
Responde con un mensaje de confirmación y un estado 201.

\begin{Json}

{
    "success": "Los datos se han insertado correctamente"
}

\end{Json}

\subsubsection{Ejemplo de Output (Error)}

\begin{Json}

\justify
Responde con un JSON incluyendo un mensaje de error.

{
  "error": "Error al añadir el usuario: (error)"
}

\end{Json}

\subsection{Eliminar un usuario}

\subsubsection{Descripción}

\justify
Elimina un registro de la tabla \texttt{users}.

\begin{itemize}

    \item \textbf{Método:} DELETE
    \item \textbf{URL:} \texttt{/api/usuarios/:id}
    
\end{itemize}

\subsubsection{Ejemplo de Input}

\justify
Parámetro \texttt{id} en la URL.

\subsubsection{Ejemplo de Output (Éxito)}

\justify
Responde con un mensaje de confirmación y un estado 201.

\begin{Json}

{
    "success": "Los datos se han insertado correctamente"
}

\end{Json}

\subsubsection{Ejemplo de Output (Error)}

\justify
Responde con un JSON incluyendo un mensaje de error.

\begin{Json}

{
  "error": "Error al añadir el usuario: (error)"
}

\end{Json}

\subsection{Cambiar el nombre de un usuario}

\subsubsection{Descripción}

\justify
Modifica el campo \textbf{name} de un registro de la tabla \texttt{users}.

\begin{itemize}

    \item \textbf{Método:} PUT
    \item \textbf{URL:} \texttt{/api/usuarios/:id}
    
\end{itemize}

\subsubsection{Ejemplo de Input}

\justify
Parámetro \texttt{id} en la URL y un objeto JSON en el body de la petición con el nombre del nuevo usuario.

\begin{Json}

{
    "name": "Adriano"
}

\end{Json}

\subsubsection{Ejemplo de Output (Éxito)}

\justify
Responde con un mensaje de confirmación y un estado 200.

\begin{Json}

{
    "success": "Se ha modificado registro correctamente"
}

\end{Json}

\subsubsection{Ejemplo de Output (Error)}

\justify
Responde con un JSON incluyendo un mensaje de error.

\begin{Json}

{
  "error": "Error al modificar el nombre del usuario: (error)"
}

\end{Json}

\subsection{Listar todos los grupos}

\subsubsection{Descripción}

\justify
Obtiene una lista de todos los registros almacenados en la tabla \texttt{groups}.

\begin{itemize}

    \item \textbf{Método:} GET
    \item \textbf{URL:} \texttt{/api/grupos}
    
\end{itemize}

\subsubsection{Ejemplo de Input}

\justify
No requiere ningún input (ni parámetros en la URL ni body).

\subsubsection{Ejemplo de Output (Éxito)}

\justify
Responde con un array de objetos JSON, cada uno representando un rapero.

\begin{Json}

[
    {
        "id": 1,
        "name": "Aulaestudienses"
    },
    {
        "id": 2,
        "name": "Antijavanianos"
    }
]

\end{Json}

\subsubsection{Ejemplo de Output (Error)}

\justify
Responde con un JSON incluyendo un mensaje de error.

\begin{Json}

{
  "error": "Error al devolver los datos de los grupos: (error)"
}

\end{Json}

\subsection{Listar un grupo en concreto}

\subsubsection{Descripción}

\justify
Obtiene un solo registro en la tabla \texttt{groups}.

\begin{itemize}

    \item \textbf{Método:} GET
    \item \textbf{URL:} \texttt{/api/grupos/:id}
    
\end{itemize}

\subsubsection{Ejemplo de Input}

\justify
Parámetro \texttt{id} en la URL.

\subsubsection{Ejemplo de Output (Éxito)}

\justify
Responde con un JSON en el que se incluye el grupo a buscar.

\begin{Json}

[
    {
        "id": 1,
        "name": "Aulaestudienses"
    }
]

\end{Json}

\subsubsection{Ejemplo de Output (Error)}

\justify
Responde con un JSON incluyendo un mensaje de error.

\begin{Json}

{
  "error": "Error al devolver los datos del usuario: (error)"
}

\end{Json}

\subsection{Añadir un grupo}

\subsubsection{Descripción}

\justify
Añade un registro a la tabla \texttt{users}.

\begin{itemize}

    \item \textbf{Método:} POST
    \item \textbf{URL:} \texttt{/api/grupos}
    
\end{itemize}

\subsubsection{Ejemplo de Input}

\justify
Un objeto JSON en el body de la petición con el nombre del nuevo grupo.

\begin{Json}

{
    "name": "Aulaestudienses"
}

\end{Json}

\subsubsection{Ejemplo de Output (Éxito)}

\justify
Responde con un mensaje de confirmación y un estado 200.

\begin{Json}

{
    "success": "Los datos se han insertado correctamente"
}

\end{Json}

\subsubsection{Ejemplo de Output (Error)}

\justify
Responde con un JSON incluyendo un mensaje de error.

\begin{Json}

{
  "error": "Error al añadir el usuario: (error)"
}

\end{Json}

\subsection{Eliminar un grupo}

\subsubsection{Descripción}

\justify
Elimina un registro de la tabla \texttt{groups}.

\begin{itemize}

    \item \textbf{Método:} DELETE
    \item \textbf{URL:} \texttt{/api/grupos/:id}
    
\end{itemize}

\subsubsection{Ejemplo de Input}

\justify
Parámetro \texttt{id} en la URL.

\subsubsection{Ejemplo de Output (Éxito)}

\justify
Responde con un mensaje de confirmación y un estado 201.

\begin{Json}

{
    "success": "Grupo eliminado correctamente"
}

\end{Json}

\subsubsection{Ejemplo de Output (Error)}

\justify
Responde con un JSON incluyendo un mensaje de error.

\begin{Json}

{
  "error": "Error al eliminar el grupo: (error)"
}

\end{Json}

\subsection{Añadir un usuario a un grupo}

\subsubsection{Descripción}

\justify
Añade un registro a la tabla \texttt{users\_groups}.

\begin{itemize}

    \item \textbf{Método:} POST
    \item \textbf{URL:} \texttt{/api/grupos/:id\_grupo/:id\_usuario}
    
\end{itemize}

\subsubsection{Ejemplo de Input}

\justify
Parámetros \texttt{id\_grupo} y \texttt{id\_usuario} en la URL.

\subsubsection{Ejemplo de Output (Éxito)}

\justify
Responde con un mensaje de confirmación y un estado 200.

\begin{Json}

{
    "success": "Los datos se han insertado correctamente"
}

\end{Json}

\subsubsection{Ejemplo de Output (Error)}

\justify
Responde con un JSON incluyendo un mensaje de error.

\begin{Json}

{
  "error": "Error al añadir el usuario al grupo: (error)"
}

\end{Json}

\subsection{Eliminar un usuario a un grupo}

\subsubsection{Descripción}

\justify
Elimina un registro de la tabla \texttt{users\_groups}.

\begin{itemize}

    \item \textbf{Método:} DELETE
    \item \textbf{URL:} \texttt{/api/grupos/:id\_grupo/:id\_usuario}
    
\end{itemize}

\subsubsection{Ejemplo de Input}

\justify
Parámetros \texttt{id\_grupo} y \texttt{id\_usuario} en la URL.

\subsubsection{Ejemplo de Output (Éxito)}

\justify
Responde con un mensaje de confirmación y un estado 200.

\begin{Json}

{
    "success": "Usuario eliminado del grupo correctamente"
}

\end{Json}

\subsubsection{Ejemplo de Output (Error)}

\justify
Responde con un JSON incluyendo un mensaje de error.

\begin{Json}

{
  "error": "Error al añadir el usuario al grupo: (error)"
}

\end{Json}

\section{ESTRUCTURA DEL PROYECTO}

\justify
En esta sección se define la estructura del proyecto:

\begin{verbatim}
    
api_usuarios_grupos/
|-- db/
|-- DOCUMENTACIÓN/
|   |-- images/
|   |-- DOCUMENTACIÓN API Y BBDD - ADRIANO.tex
|   |-- DOCUMENTACIÓN API Y BBDD - ADRIANO.pdf
|-- .dockerignore
|-- .env
|-- docker-compose.yaml
|-- Dockerfile
|-- index.js
|-- package-lock.json
|-- package.json

\end{verbatim}

\section{PROCESO DE DOCKERIZACIÓN}

\section{CI/CD CON GITHUB ACTIONS}

\end{document}